\begin{block}{Methods \& Data}
  \begin{figure}[ht]
    \centerline{\includegraphics[width=0.95\textwidth]{figs/GCM_CPC_grids.pdf}}
    \caption{
      Map of study area and Geophysical Fluid Dynamics Laboratory CM3 coupled model cells (blue grid).
      The Climate Prediction Center precipitation data (red grid) was upscaled to match the grid of the CM3 coupled model (blue grid), by taking the spatial average of the red cells inside each of the blue cells.
      The shaded area indicates the Ohio River Basin ($\sim$ 530 000 km$^2$) as defined by the United States Geological Survey.
    }
    \label{fig:area-map}
  \end{figure}
  \begin{itemize}
    \item MAM season in the Ohio River Basin (\Cref{fig:area-map}).
    ``Historical'' from 01 March 1950 through 30 May 2005, ``future'' study period from 01 March 2006 through 30 May 2100.
    \item ``Dynamical Model'' referes to GFDL global coupled model \citep{Donner2011} (CM3).
    \item CPC US unified gauge-based precipitation data, upscaled from \SI{0.25}{\degree} by \SI{0.25}{\degree} to match CM3 (\SI{2.5}{\degree} longitude by \SI{2.0}{\degree} latitude).
    \item Regional Intense Precipitation (RIP) days were defined as any day when over 20\% of the region receives experiences a 99th percentile exceedance of precipitation.
    \item Reanalysis data from NCEP/NCAR Reanalysis I \citep{Kalnay1996}.
  \end{itemize}
\end{block}
