\begin{block}{Research Questions}
  We hypothesize that GCMs may simulate the frequency and intensity  of atmospheric circulation patterns associated with regional intense precipitation events better than the statistics of the precipitation events themselves.
  \begin{description}
    \item[Q1] For the Ohio River basin, are the intense springtime precipitation events relevant for extreme floods well simulated by the GCM? If not, why?
    \item[Q2] If no to \textbf{Q1}, can we find atmospheric indices that are associated with the onset of the regional intense precipitation events?
    \item[Q3] If they are not, are suitably derived atmospheric indices associated with such events in atmospheric re-analysis relatively well simulated by the GCM?
    \item[Q4] If the GCM simulations of atmospheric indices are better, can they provide more credible projections of regional intense precipitation occurrence than the GCM can directly?
  \end{description}
\end{block}
