\begin{block}{Defining Indices for RIP Circulations}
  \begin{figure}[ht]
    \centerline{\includegraphics[width=0.75\textwidth]{figs/dipole_MHC_box_and_time_MAM.pdf}}
    \caption{The regions of low and high pressure defining the Eastern U.S.-Western Atlantic Dipole index (EWD), and the region defining the moisture holding capacity index.}
    \label{fig:EWD-MHC-location-time}
  \end{figure}
  \begin{itemize}
    \item Define Eastern U.S.-Western Atlantic Dipole (EWD) index: average of high pressure box (red) minus low pressure box (green) (\cref{fig:EWD-MHC-location-time})
    \item Define moisture-holding capacity (MHC) index using average temperature (brown; \cref{fig:EWD-MHC-location-time})
      $\text{MHC}_t = 6.1 \exp \qty[\frac{17.6T'_{t}}{T'_{t} + 243}]$
    \item CM3 reasonably simulates the distributional (\cref{fig:cdf-comp}) and persistence (not shown) features of two atmospheric indices that modulate the likelihood of RIP events
  \end{itemize}
  \begin{figure}[ht]
    \centerline{\includegraphics[width=0.95\textwidth]{figs/EWD_MHC_cdf_comp_MAM.pdf}}
    \caption{
      (Left) Cumulative distribution function for the reanalysis based dipole (solid line) and the GFDL GCM ensemble members based dipole (dashed lines) for MAM.
      (Middle) The positive tail of (Left).
      (Right) Same as (Left) for the moisture holding capacity.
    }
    \label{fig:cdf-comp}
  \end{figure}
\end{block}
