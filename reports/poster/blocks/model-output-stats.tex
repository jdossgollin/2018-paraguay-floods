\begin{block}{Model Output Statistics}
    We explore whether using \gls{mos}~\cite{Glahn:1972iw} can improve the modeled representation of rainfall (\cref{fig:subs-prob-fcst}).
    Specifically, we use: the raw model output (Raw); \gls{xlr}; \gls{hxlr}; \gls{pcr}; and \gls{cca} using 20 years of ECMWF forecasts.
    \Cref{fig:subs-prob-fcst} indicates that better forecasts are obtained when both magnitude and spatial corrections are performed (\gls{pcr} and \gls{cca}).
    The enhanced skill is achieved through the spatial corrections via the EOF-based regressions, which – in contrast with the extended logistic models – use information from multiple grid-boxes,.
    \begin{framed}
        \begin{figure}
            \noindent\includegraphics[width=0.9\textwidth]{mos_forecasts.pdf}
            \caption{
                \Gls{mos}-adjusted S2S model forecasts and skill scores.
                Top row shows the heavy rainfall ($>$\nth{90} percentile exceedance) forecast for 1-7 December 2015 as the odds ratio relative to climatology $\text{odds} = \frac{p}{1-p}\frac{1-p_c}{p_c}$.
                Second row shows the Ignorance score $\text{IGN}=-\log_2 p(Y)$.
                Bottom row shows the 2AFC skill score for each grid cell.
                For all three rows, the grid cells which experienced a \nth{90} percentile exceedance for 1-7 December 2015 are outlined in black.
            }\label{fig:subs-prob-fcst}
        \end{figure}
    \end{framed}
\end{block}
