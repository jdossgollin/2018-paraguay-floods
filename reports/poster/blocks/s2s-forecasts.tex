\begin{block}{S2S Model Forecasts}
    \begin{framed}
        \begin{figure}
            \centering
            \caption{
                Chiclet diagram of ensemble-mean precipitation anomaly forecasts over the \gls{lprb} from ECMWF S2S forecast data, as a function of the forecast target date (horizontal axis) and lead time (vertical axis).
                Time series of daily precipitation over the same area is plotted with $y$-axis inverted.
            }\label{fig:chiclet}
            \noindent\includegraphics[width=\textwidth]{chiclet.pdf}
        \end{figure}
    \end{framed}
    \Cref{fig:chiclet} uses a Chiclet diagram~\cite{Carbin:2016fx} to visualize, as a function of lead time, the time evolution of the uncorrected, ensemble-mean rainfall anomaly forecast, spatially averaged over the \gls{lprb}.
    At lead times greater than about two weeks, the ensemble-mean forecast is slightly wetter than climatology.
    At weather timescales (less than one week), the ensemble-mean successfully predicts the timing and amplitude of the area-averaged rainfall.
    At intermediate timescales, the model successfully forecast the strongest breaks and pauses in the rainfall, such as the heavy rainfall during December 2015 and the dry period during mid-January 2016.
\end{block}
