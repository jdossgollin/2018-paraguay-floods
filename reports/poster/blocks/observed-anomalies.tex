\begin{block}{Observed Circulation Anomalies}
    \begin{framed}
        \begin{figure}
            \centering
            \caption{
                Time series of area-averaged rainfall in the \acrlong{lprb} (\cref{fig:study-area}) for each day of NDJF  2015-16.
                Lines indicate the rainfall value, in units of \si{\milli\meter\per\day}.
                The weather type corresponding to each day is indicated in an adjacent text label.
                Dashed lines blue indicate (from bottom to top) the climatological 50th, \nth{90}, and 99th percentiles of NDJF area-averaged rain over the \acrlong{lprb}.
            }
            \noindent\includegraphics[width=0.9\textwidth]{rain_wt_201516.pdf}
            \label{fig:rain-wtype}
        \end{figure}
    \end{framed}
    During austral summer (NDJF) 2015-16, most heavy rainfall occurred during \glspl{wt} 1 and 4 (\cref{fig:rain-wtype}). 
    Monthly-scale circulation anomalies (\cref{fig:anomaly-ndjf}) show a weak anticyclonic circulation that set up over central Brazil during November 2015 and strengthened into the following month.
    In January 2016 it weakened before returning in February 2016.
    The observed rainfall and circulation anomalies are consistent with the aggregation of the observed weather types shown in \cref{fig:rain-wtype}.
    \begin{framed}
        \begin{figure}
            \centering
            \noindent\includegraphics[width=0.9\textwidth]{anomalies_ndjf1516.pdf}
            \caption{
                Monthly composite anomalies observed during NDJF 2015-16.
                Top row shows streamfunction anomalies at \SI{850}{\hecto\pascal}.
                Bottom row shows rainfall anomalies, in units of \si{\milli\meter\per\day}.
            }
            \label{fig:anomaly-ndjf}
        \end{figure}
    \end{framed}
\end{block}
